The raytracer project is an implementation of a rendering engine that utilizes the raytracing technique to generate realistic images by simulating the behavior of light. Raytracing is a powerful method used in computer graphics to produce highly realistic renders by tracing the path of light rays through a virtual scene.

This project, implemented in C++, encompasses a raytracing rendering engine that allows for the creation of images by calculating the interactions between light and objects in the scene. It employs optical principles such as reflection, refraction, and light diffusion to simulate the effects of light on object surfaces.

Key features of the project include 3D geometry modeling, application of textures and materials to objects, implementation of light sources, shadow handling, simulation of reflection and refraction phenomena, as well as management of depth of field and motion blur effects. It may also encompass advanced techniques such as global illumination computation and simulation of complex optical phenomena like caustics.

The raytracer project requires a solid understanding of mathematics, including geometry, linear algebra, and vector calculus. It is implemented in C++ to provide a high-\/performance and flexible environment for developing the raytracing engine.

Ultimately, the raytracing rendering engine produces realistic and compelling images by employing intricate calculations to simulate the behavior of light. It finds applications in numerous fields, such as cinematic animation, visual effects, video games, architectural design, and virtual reality, to create immersive and visually stunning virtual environments. 